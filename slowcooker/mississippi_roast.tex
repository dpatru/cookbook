\section{Mississippi Roast}
\begin{recipe}
	\source{Robin Chapman / NYTimes Cooking}
	\yield{6-8 servings}
	\cooktime{6\sfrac{1}{2} to 8\sfrac{1}{2} hours}
	\pre{This legendary pot roast contains just 5 ingredients. The NYTimes variant, second, uses several more.}
	
	
	\ingredients{
		3 to 4			& \lb chuck roast or top/bottom round roast \\
		\sfrac{1}{4}	& \cup unsalted butter \\
		5				&  pepperoncini \\
		1 				& 1 oz. ranch dressing mix \\
		1 				& 1 oz. dry au jus mix \\
	}

	Place roast in a slow cooker. Form a pocket in the top of roast and place butter, pepperoncini peppers, ranch dressing mix, and au jus mix in the pocket. Cook on low for 8 hours.

	\textbf{NYTimes variant:}
	
	\ingredients{
		3 to 4			& \lb chuck roast or top/bottom round roast \\
		2 				& \t kosher salt, plus more to taste \\
		1\sfrac{1}{2} 	& \t ground black pepper, plus more to taste \\
		\sfrac{1}{4} 	& \cup all-purpose flour \\
		3 				& \T neutral oil, like canola \\
		4 				& \T unsalted butter \\
		8 to 12			&  pepperoncini \\
		2 				& \T mayonnaise \\
		2 				& \t apple cider vinegar \\
		\sfrac{1}{2} 	& \t dried dill \\
		\sfrac{1}{4} 	& \t sweet paprika \\
		1 				& \t buttermilk, optional \\
						& chopped parsley, for garnish \\
	}
	
	
	Place roast on cutting board and rub the salt and pepper generously. Sprinkle flour over the meat and massage it in.
	
	Heat oil in a large saut\'{e} pan over high heat until it is shimmering and about to smoke. Place the roast in the pan and brown on all sides, 4 to 5 minutes a side, to create a crust. Remove roast from pan and place it in the bowl of a slow cooker. Add the butter and the pepperoncini to the meat. Put lid on slow cooker and set to low.
	
	As the roast heats, make a ranch dressing. Combine the mayonnaise, vinegar, dill and paprika in a small bowl and whisk to emulsify. Add the buttermilk if using, then whisk again. Remove the lid from slow cooker and add the dressing. Replace lid and continue cooking, undisturbed, for 6 to 8 hours, or until you can shred the meat easily using 2 forks. Mix the meat with juice, garnish with parsley, and serve with mashed potatoes, noodles, roast potatoes, or on sandwich rolls.
	
	
\end{recipe}
